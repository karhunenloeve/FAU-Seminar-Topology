\documentclass{article}
\usepackage[ngerman]{babel}
\usepackage{a4wide}
\usepackage{tikz}
\usepackage{tikz-cd}
\usetikzlibrary{babel}
\usepackage{xcolor}
\usepackage{amsmath,amsfonts,amssymb,amsthm,epsfig,epstopdf,titling,url,array}
\usepackage{url}

\theoremstyle{plain}
\newtheorem{thm}{Satz}[section]
\newtheorem{lem}[thm]{Lemma}
\newtheorem{prop}[thm]{Proposition}
\newtheorem*{cor}{Korollar}

\theoremstyle{definition}
\newtheorem{defn}{Definition}[section]
\newtheorem{conj}{Behauptung}[section]
\newtheorem{exmp}{Beispiel}[section]

\theoremstyle{remark}
\newtheorem*{rem}{Anmerkung}
\newtheorem*{note}{Note}

\usepackage{enumitem}
\setlist[itemize]{noitemsep, nolistsep}

\title{Natürliche Transformationen, Äquivalenzen von Kategorien, darstellbare Funktoren und das Lemma von Yoneda}
\author{Luciano Melodia}
\date{25.10.2022}

\begin{document}

\maketitle

\begin{abstract}
In diesem Skript behandeln wir die Definition von natürlichen Transformationen. Wir veranschaulichen natürliche Transformationen durch möglichst viele Beispiele, insbesondere solche, die für die Topologie von besonderem Interesse sind. Dann befassen wir uns mit der Frage, wann Kategorien als äquivalent angesehen werden können. Schließlich klären wir den Begriff der Darstellbarkeit für kovariante und kontravariante Funktoren.

Mit Hilfe dieser Begriffe wird es uns gelingen, das Lemma von Yoneda zu beweisen, das oft als das wichtigste mathematische Ergebnis innerhalb der Kategorientheorie angesehen wird. Es beschreibt die Menge der natürlichen Transformationen zwischen einem $\textbf{Hom}$-Funktor und einem beliebigen anderen Funktor.
\end{abstract}

\section{Natürliche Transformationen}
Im Folgenden seien $\mathcal{A},\mathcal{B}$ Kategorien, und $F,G: \mathcal{A} \rightarrow \mathcal{B}$ Funktoren. Um natürliche Transformationen definieren zu können, brauchen wir die folgenden Daten und Eigenschaften:
\begin{itemize}
	\item Einen Morphismus $\tau_{X}: F(X) \rightarrow G(X)$ für jedes Objekt $X \in \mathcal{A}$.
	\item Die Eigenschaft falls $f: X \rightarrow Y$ ein Morphismus in $\mathcal{A}$ ist, so gilt
	\begin{equation}
		G(f) \circ \tau_{X} = \tau_{Y} \circ F(f).
	\end{equation}
\end{itemize}
Dies führt uns zu folgender Definition.

\begin{defn}[Natürliche Transformationen]
Eine \textbf{natürliche Transformation} $\tau: F \Longrightarrow G$ von einem Funktor $F$ zu einem zweiten Funktor $G$ ordnet jedem Objekt $X \in \mathcal{A}$ einen Homomorphismus $\tau_{X}: F(X) \rightarrow G(X)$ in $\mathcal{B}$ zu, auch \textbf{Komponente von $\tau$ bei $X$} genannt, so dass für jeden Morphismus $f: X \rightarrow Y$ in $\mathcal{A}$ das folgende Diagramm kommutiert:
\end{defn}
\begin{equation}
	\begin{tikzcd}
	F(X) \arrow[rr, "F(f)" description] \arrow[d, "\tau_{X}" description] &  & F(Y) \arrow[d, "\tau_{Y}" description] \\
	G(X) \arrow[rr, "G(f)" description]                                   &  & G(Y)                                  
	\end{tikzcd}.
\end{equation}
Als Formel lässt sich das Diagramm ausdrücken durch folgende Gleichung: $\tau_{Y} \circ F(f) = G(f) \circ \tau_{X}$. Für Funktoren $F,G$ wird mit $\text{Mor}_{\text{Fun}}(F,G)$ die \textbf{Menge der natürlichen Transformationen} von $F$ nach $G$ bezeichnet.

\begin{rem}
Die natürliche Transformation $\tau$ ist die Gesamtheit aller Morphismen $\tau_{X}$. In der Literatur findet sich diese Tatsache in der Notation $\tau = (\tau_{X})_{X \in \mathcal{A}}$ wieder, wobei jedes $\tau_{X}$ eine Komponente von $\tau$ ist. Man kann eine gewisse Analogie zu dem Aufbau von Folgen als Gesamtheit ihrer Terme $s = (s_n)_{n \in \mathbb{N}}$ sehen.
\end{rem}

In anderen Worten ist eine natürliche Transformation eine Sammlung von Abbildungen von einem Diagramm in ein anderes. Das Besondere an diesen Abbildungen ist, dass sie kommutieren. Nehmen wir eine natürliche Transformation zwischen zwei Funktoren $\tau: F \Longrightarrow G$, dann können wir durch folgendes Diagramm eine bessere Intuition für natürliche Transformationen gewinnen:
\begin{equation}
	\begin{tikzcd}
	                                                   & {\color{red}F} \arrow[r, Rightarrow, "\tau"]                 & {\color{blue}G}, &                            &                 \\
	{\color{red}F(X)} \arrow[dd, red] \arrow[rd, red] \arrow[rrr, "\tau_{X}"] &                                         &   & {\color{blue}G(X)} \arrow[rd, blue] \arrow[dd, blue] &                 \\
	                                                   & {\color{red}F(Z)} \arrow[ld, red] \arrow[rrr, "\tau_{Z}", gray] &   &                            & {\color{blue}G(Z)} \arrow[ld, blue] \\
	{\color{red}F(Y)} \arrow[rrr, "\tau_{Y}"]                       &                                         &   & {\color{blue}G(Y)}                       &                
	\end{tikzcd}.
\end{equation}
Um das obige Diagramm in Gänze verstehen zu können, betrachten wir im Anschluss ein paar Sonderfälle und Beispiele für natürliche Transformationen.

\begin{exmp}
\begin{enumerate}
	\item Die Funktoren $F,G: \mathcal{A} \rightarrow \mathcal{B}$ seien beide konstante Funktoren, also wird jedes Objekt $X$ in $\mathcal{A}$ auf ein einziges Objekt $F(X)$ in $\mathcal{B}$ abgebildet und jeder Morphismus auf $\text{Id}_{F(Y)}$. Analog soll $G$ ein Objekt und einen Morphismus auf ein festes $F(Y)$ und $\text{Id}_{F(Y)}$ in $\mathcal{B}$ abbilden. Dann ist eine natürliche Transformation von $F$ nach $G$ gegeben durch $\tau: F(X) \Longrightarrow F(Y)$.
	\item Sei $F: \mathcal{A} \rightarrow \mathcal{B}$ konstant für ein Objekt $X$ in $\mathcal{A}$ und $G: \mathcal{A} \rightarrow \mathcal{B}$ ein beliebiger Funktor. Dann besteht die natürliche Transformation $\tau: F \Longrightarrow G$ aus Abbildungen $\tau_{X}: F(X) \rightarrow G(X)$, eine für jedes $X \in \mathcal{A}$, sodass $\tau_Y = G(f) \circ \tau_X$, falls $f:X \rightarrow Y$ ein Morphismus in $\mathcal{A}$ ist. Veranschaulichen kann man das durch folgendes kommutierende Diagram, 
	\begin{equation}
		\begin{tikzcd}
		               & F(X) \arrow[lddd, "\tau_X" description, bend right] \arrow[dd, "\tau_V" description] \arrow[rdd, "\tau_W" description, bend left] \arrow[ddd, "\tau_Y" description, bend left] &                 \\
		               &                                                                                                                                                                             &                 \\
		               & G(V) \arrow[r] \arrow[ld]                                                                                                                                                   & G(W) \arrow[ld] \\
		G(X) \arrow[r, "G(f)"] & G(Y)                                                                                                                                                                        &                
		\end{tikzcd},
	\end{equation}
	wobei $X,Y,V,W \in \mathcal{A}$ sind und die Knoten und Kanten des unteren Vierecks das durch $G$ gegebene Diagramm meint. Die Gleichung $\tau_Y = G(f) \circ \tau_X$, die erfüllt werden muss, bringt jeweils die Dreiecke im Tetraeder zum kommutieren. In diesem Fall heißt $\tau$ \textbf{Kegel über $G$}.
	\item Sei andererseits $G$ konstant bei $X \in \mathcal{A}$, dann besteht eine natürliche Transformation aus einer Familie von Abbildungen $\tau_X: F(X) \rightarrow G(Y)$, sodass $\tau_Y \circ F(f) = \tau_X$ gilt, genau dann wenn $f: X \rightarrow Y$ ein Morphismus in $\mathcal{A}$ ist. Veranschaulicht durch ein kommutierendes Diagramm,
	\begin{equation}
		\begin{tikzcd}
		                                                         & F(V) \arrow[ld] \arrow[r] \arrow[ddd, "\tau_V" description, bend right, shift right] & F(W) \arrow[ld] \arrow[lddd, "\tau_W"', bend left] \\
		F(X) \arrow[r, "F(f)"] \arrow[rdd, "\tau_X", bend right] & F(Y) \arrow[dd, "\tau_Y"', bend left]                                                &                                                    \\
		                                                         &                                                                                      &                                                    \\
		                                                         & G(X)                                                                                 &                                                   
		\end{tikzcd}
	\end{equation}
	müssen die Seiten des Diagramms kommutieren. Dieses Diagramm heißt \textbf{Kegel unter $F$}, oder manchmal auch \textbf{Kokegel}.
	\item Seien $F,G: \mathcal{A} \rightarrow \mathcal{B}$ Funktoren und $f: X \rightarrow Y$ ein Morphismus in $\mathcal{A}$. Falls jede Kompontente $\tau_X: F(X) \rightarrow G(X)$ von $\tau$ ein Isomorphismus ist, dann ist die Bedingung $\tau_Y \circ F(f) = G(f) \circ \tau_X$ äquivalent zu $F(f) = \tau_y^{-1} \circ G(f) \circ \tau_X$, da $\tau_Y$ invertierbar ist. In einer Kategorie ist der Isomorphismusbegriff so definiert, dass für einen Morphismus $f:X \rightarrow Y$ und $g:Y \rightarrow X$ gilt $g \circ f = \text{Id}_X$ und $f \circ g = \text{Id}_Y$, genau dann wenn $X$ und $Y$ isomorph sind. In einem kommutierenden Diagramm lässt sich das wie folgt darstellen:
	\begin{equation}
		\begin{tikzcd}
		{\tau_Y \circ F(f) = G(f) \circ \tau_X,}   &                          &  & F(f) = \tau_y^{-1} \circ G(f) \circ \tau_X,            &                                \\
		F(X) \arrow[r, "F(f)"] \arrow[d, "\tau_X"] & F(Y) \arrow[d, "\tau_Y"] &  & F(X) \arrow[r, "\tau_y^{-1} \circ G(f) \circ \tau_X"] & F(Y)                           \\
		G(X) \arrow[r, "G(f)"]                     & {G(Y),}                  &  & G(X) \arrow[u, "\tau_X^{-1}"] \arrow[r, "G(f)"]       & G(Y). \arrow[u, "\tau_Y^{-1}"]
		\end{tikzcd}
	\end{equation}
	Wenn also jedes $\tau_X$ ein Isomorphismus ist, ist die Natürlichkeitsbedingung vergleichbar mit der Konjugation. Sie erinnert auch an eine Homotopie von $G$ nach $F$. Beide Sichtweisen legen nahe, dass, wenn jedes $\tau_X$ ein Isomorphismus ist, $F$ und $G$ bis auf Umbenennung ihrer Elemente wirklich derselbe Funktor sind. Wenn dies der Fall ist, wird die natürliche Transformation $\tau$ als \textbf{natürlicher Isomorphismus} bezeichnet, und $F$ und $G$ werden als \textbf{natürlich isomorph} bezeichnet.
\end{enumerate}
\end{exmp}

\subsection{Die Kategorie $\textbf{B}G$ der Gruppen}
Wir betrachten eine Gruppe $G$.

\begin{prop}
Dann ist $\textbf{B}G$ mit einem Objekt $G$ und einem Morphismus $g: G \rightarrow G$ für jedes Element $g \in G$ eine Kategorie.
\end{prop}

\begin{proof} Da eine Gruppe abgeschlossen ist unter der Komposition, gilt für zwei Morphismen $g,h: G \rightarrow G$, welche kompatibel zueinander sind, dass auch $g \circ h : G \rightarrow G$ ein Morphismus ist, nämlich das Gruppenelement $gh \in G$. Kompatibilität meint, dass das Bild von $h$ gleich dem Urbild von $g$ ist, $\text{Im}(h) = g^{-1}$. Die Identität $e \in G$ fungiert als Identitätsmorphismus für $G$ und die Assoziativität gilt aufgrund dessen, da Gruppenoperationen bereits assoziativ sind.
\end{proof}

Nun können wir einen Funktor $F: \textbf{B}G \rightarrow \textbf{Set}$ in die Kategorie der Mengen definieren, der jedes Objekt $G \in \textbf{B}G$ auf genau eine Menge $M \in \textbf{Set}$ abbildet, und jedes Gruppenelement $g \in G$ auf eine Funktion $g \cdot - : M \rightarrow M, m \mapsto g \cdot m$ abbildet.

\begin{prop}
Die Funktorialität determiniert für jeden Funktor eine linke Gruppenoperation und das Bild eines einzelnen Objekts unter diesem Funktor ist eine $G$-Menge.
\end{prop}

\begin{proof}
Seien $C,D: \textbf{B}G \rightarrow \textbf{Set}$ Funktoren, sodass $C(G) = M$ und $D(G) = N$ für $M,N \in \textbf{Set}$ und sei $g: G \rightarrow G$ ein Element der Gruppe. Dann ist $\tau: C \Longrightarrow D$ eine natürliche Transformation, bestehend aus genau einer Funktion $\tau: M \rightarrow N$, sodass $\tau(g(m)) = g(\tau(m))$, für jedes $m \in M$.
\end{proof}

Natürliche Transformationen sind also \textbf{$G$-äquivariante Abbildungen}. Diese Gleichung lässt sich mit dem nachfolgenden kommutierenden Diagramm noch einmal veranschaulichen.
	\begin{equation}
		\begin{tikzcd}
		M \arrow[rrr, "\tau" description] \arrow[ddd, "g\cdot-" description] &                                         &                            & N \arrow[ddd, "g\cdot-" description] \\
		                                                                     & m \arrow[d, maps to] \arrow[r, maps to] & \tau(m) \arrow[d, maps to] &                                      \\
		                                                                     & gm \arrow[r, maps to]                   & \tau(g(m)) = g(\tau(m))    &                                      \\
		M \arrow[rrr, "\tau" description]                                    &                                         &                            & N.                                   
		\end{tikzcd}
	\end{equation}

Ohne Zweifel ist man bereits über den Begriff des (Co)limes in der Kategorientheorie gestoßen. In den obigen Beispielen der Kegel und Kokegel werden bereits einige (Co)limites beschrieben. Beispiele wären die leere Menge, die Einpunktmenge, der Schnitt, die Vereinigung, das Produkt von Mengen, der Kern einer Gruppe, der Quotient von topologischen Räumen, die direkte Summe zweier Vektorräume, das freie Produkt von Gruppen oder der Pullback über ein Faserbündel. Jedes ist ein Spezialfall, welches eine Art universellen Kegel über einen bestimmten Funktor bildet.

\subsection{Die Kategorie $\textbf{Vect}_{\mathbb{K}}$ der $\mathbb{K}$-Vektorräume}
Wir betrachten nun ein typisches Beispiel für natürliche Transformtionen, für das wir Beweisen wollen, dass es sich wirklich um eine natürliche Transformation handelt. Sei $V$ dafür ein endlich dimensionaler Vektorraum über einem Körper $\mathbb{K}$, dann ist dieser isomorph zum Dualraum $V^*$ sowie isomorph zum Doppeldualraum $V^{**}$. Beide sind Vektorräume über $\mathbb{K}$:
\begin{align}
	V^* = \{ \text{lineare Abbildungen} \ V \rightarrow \mathbb{K} \}, \\
	V^{**} = \{ \text{lineare Abbildungen} \ V^* \rightarrow \mathbb{K} \}.
\end{align}
Im ersten Fall gilt, falls $\{v_1, \ldots, v_n\}$ eine Basis von $V$ ist, dass auch $\{v_1^*, \ldots, v_n^*\}$ eine Basis von $V^*$ ist, wobei für jedes $i \in \{1, \ldots, n\}$ die Abbildung $v_i^*: V \rightarrow \mathbb{K}$ gegeben ist durch
\begin{equation}
	v_i^*(v_j) = \begin{cases}
	1, \quad \text{if} \ i = j, \\
	0, \quad \text{if} \ i \neq j.
	\end{cases}
\end{equation}
Wie bereits aus der linearen Algebra bekannt ist, handelt es sich bei $V \rightarrow V^*$ nicht um einen kanonischen Isomorphismus. Unterschiedliche Basen erzeugen unterschiedliche Isomorphismen. Man kann sogar keinen Isomorphismus angeben, ohne sich vorher auf eine Basis festzulegen. Andererseits gibt es einen Isomorphismus von $V \rightarrow V^{**}$, der keine Wahl einer Basis benötigt: für jeden Vektor $v \in V$ und ein Element $f: V \rightarrow \mathbb{K}$, also $f \in V^*$, sei $\text{eval}_v(f) := f(v)$ die Evaluationsabbildung. Diese Abbildung wird in der Literatur häufig als \textbf{natürlicher Isomorphismus} bezeichnet. Was lässt sich für solche Isomorphismen also Schlussfolgern?

\begin{prop}
Die Gesamtheit aller Evaluationsabbildungen ist eine natürliche Transformation zwischen zwei Funktoren.
\end{prop}

\begin{proof}
Um das zu sehen, sei $(-)^{**}: \textbf{Vect}_{\mathbb{K}} \rightarrow \textbf{Vect}_{\mathbb{K}}$ der doppelte Dualfunktor, der jeden Vektorraum $V$ auf seinen Doppeldualraum $V^{**}$ abbildet und die linearen Abbildungen $\phi: V \rightarrow W$ auf $\phi^{**}: V^{**} \rightarrow W^{**}$.
\end{proof}

\section{Äquivalenzen von Kategorien}
Äquivalente Kategorien sind identisch, bis auf die Tatsache, dass sie eine unterschiedliche Anzahl isomorpher Kopien eines Objekts enthalten können. Diese Aussage wollen wir in diesem Abschnitt präzisieren, indem wir definieren wann zwei Kategorien als äquivalent gelten. Seien dazu $\mathcal{A}, \mathcal{B}$ Kategorien. Man könnte die Äquivalenz von Kategorien wie folgt definieren:

\begin{defn}[Äquivalente Kategorien]
Die Kategorien $\mathcal{A}$ und $\mathcal{B}$ heißen \textbf{äquivalent}, falls es Funktoren $F: \mathcal{A} \rightarrow \mathcal{B}$ und $G: \mathcal{B} \rightarrow \mathcal{A}$ gibt, so dass der Funktor $F \circ G$ äquivalent ist zu $\text{Id}_{\mathcal{B}}$ und der Funktor $G \circ F$ äquivalent ist zu $\text{Id}_\mathcal{A}$.
\end{defn}

Diese Definition ist jedoch nicht besonders nützlich. Es ist nicht praktikabel etwas wie eine Inverse von Funktoren zu definieren, da das Auffinden einer solchen Inverse sich meist als äußerst schwierig herausstellt. Die folgende Definition schafft daher Abhilfe:

\begin{defn}[Treue und volltreue Funktoren]
Ein Funktor $F: \mathcal{A} \rightarrow \mathcal{B}$ heißt
\begin{enumerate}
	\item \textbf{treu}, falls für alle Objekte $X,Y \in \mathcal{A}$ die durch $F$ induzierte Abbildung $\text{Mor}_\mathcal{A}(X,Y) \rightarrow \text{Mor}_\mathcal{B}(F(X),F(Y))$ injektiv ist.
	\item \textbf{volltreu}, falls für alle Objekte $X$ und $Y$ in $\mathcal{A}$ die durch $F$ induzierte Abbildung $\text{Mor}_\mathcal{A}(X,Y) \rightarrow \text{Mor}_\mathcal{B}(F(X),F(Y))$ eine Bijektion ist.
\end{enumerate}
\end{defn}

\begin{defn}[Äquivalenz von Kategorien]
Ein Funktor $F: \mathcal{A} \rightarrow \mathcal{B}$ heißt \textbf{Äquivalenz von Kategorien}, falls er volltreu ist und surjektiv auf Isomorphieklassen, wenn es also für jedes $Y \in \mathcal{B}$ ein $X \in \mathcal{A}$ gibt mit $F(X) \cong Y$. Die Kategorien $\mathcal{A}$ und $\mathcal{B}$ heißen \textbf{äquivalent}, wenn es eine Äquivalenz $F: \mathcal{A} \rightarrow \mathcal{B}$ gibt.
\end{defn}

\begin{exmp}
Sei $\mathcal{A}$ die Kategorie mit nur einem Objekt $X$ und nur einem Morphismus $\text{Id}_X: X \rightarrow X$. Sei $\mathcal{B}$ die Kategorie mit genau zwei Objekten $\widetilde{X}, \widetilde{Y}$ und den Morphismen $\text{Id}_{\widetilde{X}}:\widetilde{X} \rightarrow \widetilde{X}, \text{Id}_{\widetilde{Y}}: \widetilde{Y} \rightarrow \widetilde{Y}, a: \widetilde{X} \rightarrow \widetilde{Y}$ und $b: \widetilde{Y} \rightarrow \widetilde{X}$ mit den Verkettungen $a \circ b = \text{Id}_{\widetilde{Y}}$ und $b \circ a = \text{Id}_{\widetilde{X}}$. 
\begin{conj}
Die Kategorien $\mathcal{A}$ und $\mathcal{B}$ sind äquivalent.
\end{conj}
\begin{proof}
Sei $F: \mathcal{A} \rightarrow \mathcal{B}$ ein Funktor, sodass $F(\text{Id}_X) = \text{Id}_{\widetilde{X}}$.
\begin{enumerate}
	\item Dann ist die durch $F$ induzierte Abbildung $\text{Mor}_\mathcal{A}(X,X) \rightarrow \text{Mor}_{\mathcal{B}}(F(X),F(X))$ injektiv, denn $\text{Id}_{\widetilde{X}} = b \circ a$. Also ist der Funktor $F$ treu.
	\item Es gilt außerdem dass die von $F$ induzierte Abbildung $\text{Mor}_\mathcal{A}(X,X) \rightarrow \text{Mor}_{\mathcal{B}}(\widetilde{X},\widetilde{X})$ surjektiv ist, denn
	\begin{equation}
		\text{Mor}_{\mathcal{B}}(\widetilde{X},\widetilde{X}) := \{\text{Id}_{\widetilde{X}}\},
	\end{equation}
	ist eine einelementige Menge. Also ist $F$ volltreu.
	\item Es bleibt zu zeigen, dass für jedes $\widetilde{X} \in \mathcal{B}$ ein $X \in \mathcal{A}$ existiert, sodass $F(X) \cong \widetilde{X}$. Sei also $F(X) = \widetilde{X}$, so gilt $F(X) \cong \widetilde{X}$ trivialerweise. Aber da $b \circ a = \text{Id}_{\widetilde{X}}$ und $a \circ b = \text{Id}_{\widetilde{Y}}$, ist $\widetilde{X} \cong \widetilde{Y}$. Weil Isomorphie eine Äquivalenzrelation auf Klassen definiert, gilt auch $F(X) \cong \widetilde{Y}$.
\end{enumerate}
Damit ist gezeigt, dass $\mathcal{A}$ und $\mathcal{B}$ äquivalent sind.
\end{proof}
\end{exmp}

\section{Darstellbare Funktoren}

\paragraph{Beispiele}

\section{Yoneda Lemma}


\begin{thebibliography}{9}
\bibitem{A} Peter Fiebig (2020) \emph{Einführung in die Darstellungstheorie. Vorlesungsskript}, Friedrich-Alexander Universität Erlangen-Nürnberg, Erlangen.
\bibitem{B} Tai-Danae Bradley, Tyler Bryson, and John Terilla (2020) \emph{Topology. A Categorical Approach}, MIT Press, Cambridge.
\bibitem{B} Tai-Danae Bradley (2017) \emph{What is a Functor? Definition and Examples, Part I + II}, URL: \url{https://www.math3ma.com/blog/what-is-a-functor-part-1}, eingesehen am 17.20.2022.
\bibitem{B} Tai-Danae Bradley (2017) \emph{What is a Natural Transformation? Definition and Examples, Part I + II}, URL: \url{https://www.math3ma.com/blog/what-is-a-natural-transformation-1}, eingesehen am 17.20.2022.
\end{thebibliography}
\end{document}
